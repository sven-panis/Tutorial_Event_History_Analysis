% Options for packages loaded elsewhere
\PassOptionsToPackage{unicode}{hyperref}
\PassOptionsToPackage{hyphens}{url}
%
\documentclass[
]{article}
\usepackage{amsmath,amssymb}
\usepackage{iftex}
\ifPDFTeX
  \usepackage[T1]{fontenc}
  \usepackage[utf8]{inputenc}
  \usepackage{textcomp} % provide euro and other symbols
\else % if luatex or xetex
  \usepackage{unicode-math} % this also loads fontspec
  \defaultfontfeatures{Scale=MatchLowercase}
  \defaultfontfeatures[\rmfamily]{Ligatures=TeX,Scale=1}
\fi
\usepackage{lmodern}
\ifPDFTeX\else
  % xetex/luatex font selection
\fi
% Use upquote if available, for straight quotes in verbatim environments
\IfFileExists{upquote.sty}{\usepackage{upquote}}{}
\IfFileExists{microtype.sty}{% use microtype if available
  \usepackage[]{microtype}
  \UseMicrotypeSet[protrusion]{basicmath} % disable protrusion for tt fonts
}{}
\makeatletter
\@ifundefined{KOMAClassName}{% if non-KOMA class
  \IfFileExists{parskip.sty}{%
    \usepackage{parskip}
  }{% else
    \setlength{\parindent}{0pt}
    \setlength{\parskip}{6pt plus 2pt minus 1pt}}
}{% if KOMA class
  \KOMAoptions{parskip=half}}
\makeatother
\usepackage{xcolor}
\usepackage[margin=1in]{geometry}
\usepackage{graphicx}
\makeatletter
\newsavebox\pandoc@box
\newcommand*\pandocbounded[1]{% scales image to fit in text height/width
  \sbox\pandoc@box{#1}%
  \Gscale@div\@tempa{\textheight}{\dimexpr\ht\pandoc@box+\dp\pandoc@box\relax}%
  \Gscale@div\@tempb{\linewidth}{\wd\pandoc@box}%
  \ifdim\@tempb\p@<\@tempa\p@\let\@tempa\@tempb\fi% select the smaller of both
  \ifdim\@tempa\p@<\p@\scalebox{\@tempa}{\usebox\pandoc@box}%
  \else\usebox{\pandoc@box}%
  \fi%
}
% Set default figure placement to htbp
\def\fps@figure{htbp}
\makeatother
\setlength{\emergencystretch}{3em} % prevent overfull lines
\providecommand{\tightlist}{%
  \setlength{\itemsep}{0pt}\setlength{\parskip}{0pt}}
\setcounter{secnumdepth}{-\maxdimen} % remove section numbering
\usepackage{framed}
\usepackage{xcolor}
\let\oldquote\quote
\let\endoldquote\endquote
\renewenvironment{quote}{\begin{leftbar}}{\end{leftbar}}
\usepackage{bookmark}
\IfFileExists{xurl.sty}{\usepackage{xurl}}{} % add URL line breaks if available
\urlstyle{same}
\hypersetup{
  pdftitle={Review 1: Author responses},
  pdfauthor={Panis \& Ramsey},
  hidelinks,
  pdfcreator={LaTeX via pandoc}}

\title{Review 1: Author responses}
\author{Panis \& Ramsey}
\date{07.07.2025}

\begin{document}
\maketitle

\section{Editor}\label{editor}

\begin{quote}
15-Mar-2025

Dear Dr.~Panis:

Thank you for submitting your Tutorial entitled ``Event History Analysis
for psychological time-to-event data: A tutorial in R with examples in
Bayesian and frequentist workflows'' . I have now received input from
three experts in the field, and you will find the reviews below this
message and attached. As you will see, all three reviewers believe your
paper can make a strong contribution to the literature. Prior to
considering the reviewers' input, I read your paper to form an
independent opinion. I share the positive sentiments about this work and
invite you to revise and resubmit your manuscript for further
consideration.
\end{quote}

\begin{quote}
The reviews are through and thoughtful, and I will not reiterate all the
details here. For my part, I would like your revision to focus on
accessibility. Your paper is a bit challenging-- it's long and a little
tough to get through. What can make it easier for less familiar readers?
Perhaps more of a basic set up earlier? A glossary defining specific
terms (e.g., what is an event? how is it defines? censoring? hazards
vs.~survival curves, conditional probabilities, etc.)? Also, clarify
when readers need to shift from frequentist to Bayesian model\ldots{}
and why? The main consideration here is that wish to make the paper much
more accessible-- from start to finish-- for readers who are novices and
entirely unfamiliar with this area. This can be done any number of ways,
and R3 also includes some suggestions for making these reviews. I
encourage you to have the revised paper read and pre-reviewed by
colleagues who are novices in this area with the explicit goal of making
the paper accessible to them-- this will help you hit the more for our
broad readership.
\end{quote}

\begin{quote}
If you choose to submit a revision, please include include a letter
detailing your point-by-point responses to each reviewer comment and
indicating how you changed the manuscript to address them. The revised
manuscript may undergo further peer review. We ask that you submit your
revision within three months. Please let us know if you will not be able
to meet this deadline.
\end{quote}

\begin{quote}
Sincerely,
\end{quote}

Response: Thank you for providing constructive feedback and allowing us
to submit a revision of our manuscript (AMPPS-24-0232).

We tried to focus on accessibility in the revision of our Tutorial in
several ways. First, we replaced Figure 1 with a simpler Figure that
directly conveys our message, and deleted Figures 2 and 3. Second, we
shortened the Introduction and Discussion sections. Third, we added a
section in the supplementary material (section A) in which we visualize
the different types of time-to-event data that are obtained in typical
RT tasks (detection, discrimination, bistable perception). We also
mention the unique contribution of our work, we removed redundancies,
and tried to improve the clarity of the paper. As such, we have
substantially revised the manuscript in accordance with feedback from
you as well as the reviewers.

\section{Reviewer: 1}\label{reviewer-1}

\begin{quote}
Comments to the Author It was a pleasure to review this manuscript as
someone who has spent years applying survival analysis methods to
psychological data. This is an ambitious manuscript and I believe it
achieves its goals. The key contributions of this manuscript from my
perspective are the creation of custom functions for data wrangling, and
the section on planning studies/simulations for power analysis (of
course, in addition to the introduction to this category of methods and
the practical tutorials provided. I have some comments on this work that
I believe could strengthen this manuscript's contribution.
\end{quote}

\begin{quote}
General comments:
\end{quote}

\begin{quote}
The Introduction is really nicely framed in terms of providing
compelling information for why this class of statistical models would be
useful in some specific areas of psychology (i.e., reaction time data).
Given that there are already a couple of tutorial-types of articles
introducing specific applications of survival analysis in developmental
psychology (references below, both regarding the use of these methods
for behavioural observations of emotional expressions in a multilevel
framework for recurring events), it would be helpful to further frame
the unique contribution that this work adds. The authors may also
consider further emphasizing the specific theoretical contributions that
using these methods may lead to in the areas of reaction time data etc.
\end{quote}

Response: In the revised ms. we frame the unique contribution that this
works adds on page 6: ``\ldots we are not aware of any tutorials that
are aimed specifically at psychological RT (+ accuracy) data, and which
provide worked examples of the key data processing and Bayesian
multilevel regression modelling steps.''

\begin{quote}
Lougheed, J. P., Benson, L., Ram, N., \& Cole, P. M. (2019). Multilevel
survival analysis: Studying the timing of children's recurring
behaviors. Developmental Psychology, 55(1), 53--65.
\url{https://doi.org/10.1037/dev0000619}
\end{quote}

\begin{quote}
Stoolmiller, M. (2016). An introduction to using multivariate multilevel
survival analysis to study coercive family process. In T. J. Dishion \&
J. J. Snyder (Eds.), The Oxford handbook of coercive relationship
dynamics (pp.~363--378). Oxford University Press.
\end{quote}

\begin{quote}
Stoolmiller, M., \& Snyder, J. (2006). Modeling heterogeneity in social
interaction processes using multilevel survival analysis. Psychological
Methods, 11(2), 164--177.
\url{https://doi.org/10.1037/1082-989X.11.2.164}
\end{quote}

Response: We added these references to the revised ms. on page 6.

\begin{quote}
Similarly, it would be helpful to make it clear that this manuscript
pertains to estimating single-event occurrences rather than recurring
events (which are described in the Lougheed and Stoolmiller tutorials
mentioned above)
\end{quote}

Response: We explicitly mention this on page 6 of the revised ms. Also,
we added a section in the Supplemental Material (section A in the
revised ms.) which explains the types of time-to-event data typically
obtained in RT tasks (detection, discrimination, and bistable perception
tasks). This section makes it clear that our tutorials pertain only to
estimating single-event occurrences. It also visualizes recurrent events
and includes references to the Lougheed and Stoolmiller tutorials.

\begin{quote}
Specific comments:
\end{quote}

\begin{quote}
The authors use several different terms to refer to time-to-event models
throughout the manuscript. The title refers to EHA, and the running head
refers to Hazard analysis. Survival analysis is also a common term in
Psychology (see tutorials by Lougheed et al., 2019, and Stoolmiller \&
Snyder, 2006). Disambiguating the various terms early in the manuscript
and then constantly using one term throughout the manuscript would be
helpful.
\end{quote}

Response: We disambiguate the various terms in the revised ms. on pages
7-8, and we now only use one term throughout.

\begin{quote}
Page 5: it would be helpful to refer to the same time unit in reference
to Figure 1, which shows time bins along the x-axis but the examples
discussed in the manuscript refer to units of milliseconds.
\end{quote}

Response: Figure 1 has been replaced.

\begin{quote}
Page 7: The authors state that they are not aware of EHA tutorials for
psychological time to event data. Previoulsy mentioned tutorial articles
by Lougheed and Stoolmiller could be mentioned here--- these are
demonstrating applications to behavioural observation data within
psychology.
\end{quote}

Response: We added these 3 references to the revised ms. on page 6.

\begin{quote}
Page 8: I'm curious about the decision to focus on discrete time rather
than continuous time models. It would help the reader to explain the
decision either way, or refer the reader to Lougheed et al., 2019 who
discuss this distinction and choices that researchers may make regarding
how to handle decisions around time given data formats. I don't work
with reaction time data, but it seems to me that many experiments in
this area would collect data in continuous time intervals, and it's not
clear to me what the benefit of discretizing into time bins is. Now that
I've read the entire manuscript, I see that some information is
presented on this on Page 50, but it would be helpful to make this
information known to the reader earlier in the manuscript.
\end{quote}

Response: We address our choice for discrete-time methods on pages 8-9
of the revised ms.:

``\ldots{} the definition of hazard and the type of models employed
depend on whether one is using continuous or discrete time units. As a
lab, and mainly for practical reasons, we have much more experience
using discrete-time EHA, and that is the approach that we describe and
focus on in this paper. This choice may seem counter-intuitive, given
that RT is typically treated as a continuous variable. However,
continuous forms of EHA require much more data to estimate the
continuous-time hazard (rate) function well (Bloxom, 1984; Luce, 1991;
Van Zandt, 2000). Thus, by trading a bit of temporal resolution for a
lower number of trials, discrete-time methods seem ideal for dealing
with typical psychological RT data sets for which there are less than
\textasciitilde200 trials per condition per participant (Panis, Schmidt,
et al., 2020). Moreover, as indicated by Allison (2010), learning
discrete-time EHA methods first will help in learning continuous-time
methods, so it seems like a good starting point.''

\begin{quote}
Page 8-9, excellent and accessible explanations of the survival function
in the discrete time framework.
\end{quote}

Response: Thank you for the positive feedback. To make the manuscript
more accessible and shorter and to respond to some other comments from
the editor and other reviewers, we moved the explanation of the survival
function to the Supplementary Material in the revised ms.

\begin{quote}
Page 11: In my experience, researchers in psychology are generally not
familiar with the concept of right censored data. This term is first
used on page 11 with no explanation--- a brief, even parenthetical,
definition would be helpful. For example, the parenthetical definition
on page 48 could be provided on page 11.
\end{quote}

Response: We introduce the definition of right-censoring on page 10 of
the revised ms.

\begin{quote}
Taken together, this is an excellent manuscript with the potential to
further bring survival analysis methods into Psychology, where
time-to-event and right-censored cases abound, yet very few people seem
to consider these issues.
\end{quote}

Thank you very much for your helpful and positive feedback.

\section{Reviewer: 2}\label{reviewer-2}

\begin{quote}
Comments to the Author Please see attached file for comments.
\end{quote}

\begin{quote}
Reviewer report for AMPPS-24-0232: ``Event History Analysis for
psychological time-to- event data: A tutorial in R with examples in
Bayesian and frequentist workflows'' Summary: The authors present a
tutorial paper on the analysis of discrete time-to-event outcomes. They
motivate their tutorials with a description of psychological experiments
that produce time-to-event data and explain how simply summarizing
event-time outcomes using means results in a loss of potentially
important information. They then explain how more information can be
gleaned from event-time data if they are summarized using event history
analysis methods. They focus specifically on the analysis of discrete
time-to-event outcomes. They provide code for conducting the analyses in
the tutorials on their Github. Overall, I like the idea of this paper
and strongly agree that using methods appropriate for analyzing
time-to-event data, rather than simply summarizing event-time outcomes
using means, is very important and currently underused. While I like the
premise of the paper and think that such a paper could be a valuable
contribution to the literature, I do think the paper has room for
improvement in a number of areas (see comments below). The writing is
approachable (which is advantageous for a tutorial paper) but some text
in the paper was repetitive and could be a bit more polished; additional
editing with a focus on clarity and brevity would be beneficial.
\end{quote}

Response: Thanks for your constructive and encouraging feedback. As we
also noted in our response to the editor, we have made substantial
revisions to the manuscript that include removing repetitive text, and
placing a stronger focus on clarity and brevity in the revised ms.

\begin{quote}
Major comments: 1. Some discussion on the choice of bin width is needed.
Since all of the example time-to-event data is collected in continuous
time, the analyst must discretize it by defining bins. Defining bins is
a key step in the analysis pipeline and so should not be overlooked. It
seems like results---and potentially conclusions drawn---may depend on
the choice of bin width. What factors should the analyst consider when
defining the width of the bins? Does the number of bins matter? Or is
the number of events per bin important? What about other factors? Should
analysts (or potentially these authors too) consider multiple bin widths
when conducting an analysis? An example comparing results when bin width
varies could be both interesting and informative.
\end{quote}

Response: We discuss the issue of defining bins on page 12 of the
revised ms.:

``Third, the width of each time bin will need to be determined. For
instance, in Figure 1B we chose 100 ms in an arbitrary manner. In
reality, however, bin width will need to be set by considering a number
of factors simultaneously. The optimal bin width will depend on (a) the
length of the observation period in each trial, (b) the rarity of event
occurrence, (c) the number of repeated measures (or trials) per
condition per participant, and (d) the shape of the hazard function.
Finding an appropriate bin width in a given user case before fitting
models will require testing a number of options, when calculating and
plotting the descriptive statistics (see section 3.1). The goal is to
find the smallest bin width that is supported by the amount of data
available. Based on our experience, a bin width of 50 ms is a good
starting value when the number of repeated measures is 100 or less. Too
small bin widths will result in erratic hazard functions as many bins
will have no events, and thus hazard estimates of zero. Interestingly,
the time bins do not need to have the same width. For example, Panis
(2020) used larger bins towards the end of the observation period, as
fewer events occurred there.''

\begin{quote}
\begin{enumerate}
\def\labelenumi{\arabic{enumi}.}
\setcounter{enumi}{1}
\tightlist
\item
  In the discussion, the authors mention that they choose to present a
  tutorial on discrete time analysis methods because they believe these
  are easier to understand. My personal impression is that awareness of
  continuous time analysis methods is slightly higher and so researchers
  may be more familiar with methods such as Cox models, even if they
  have not used them themselves; however, this could vary by specific
  sub-field. I agree with the authors that the interpretation of the
  hazard is easier in the discrete time setting than in the continuous
  time setting, but I do think that the distinction between the
  interpretation of the hazard in continuous vs discrete time should be
  discussed in the introduction. When I first read the introduction, I
  was confused as to why the focus was on discrete time methods, when
  the outcome measures collected in the motivating data were collected
  in continuous time. I would suggest either of the following: (a)
  provide concrete examples of when truly discrete timeoutcomes are
  collected to strengthen the motivation of the paper and justification
  for use of discrete time analysis methods or (b) keep the motivating
  examples as- is and include a short comparison of discrete
  vs.~continuous time analysis methods to help explain your reasoning
  for choosing to focus on discrete time analysis methods.
\end{enumerate}
\end{quote}

Response: We address our choice for discrete-time methods on pages 8-9
of the revised ms.:

``\ldots{} the definition of hazard and the type of models employed
depend on whether one is using continuous or discrete time units. As a
lab, and mainly for practical reasons, we have much more experience
using discrete-time EHA, and that is the approach that we describe and
focus on in this paper. This choice may seem counter-intuitive, given
that RT is typically treated as a continuous variable. However,
continuous forms of EHA require much more data to estimate the
continuous-time hazard (rate) function well (Bloxom, 1984; Luce, 1991;
Van Zandt, 2000). Thus, by trading a bit of temporal resolution for a
lower number of trials, discrete-time methods seem ideal for dealing
with typical psychological RT data sets for which there are less than
\textasciitilde200 trials per condition per participant (Panis, Schmidt,
et al., 2020). Moreover, as indicated by Allison (2010), learning
discrete-time EHA methods first will help in learning continuous-time
methods, so it seems like a good starting point.''

\begin{quote}
\begin{enumerate}
\def\labelenumi{\arabic{enumi}.}
\setcounter{enumi}{2}
\tightlist
\item
  Censoring is an important part of time-to-event data and so more
  description of censoring---and how it is (not) accounted for in
  ``orthodox'' analysis methods vs. discrete time-to-event analysis
  methods would be useful, beyond the already- included mention of
  censoring in Section 5.3.
\end{enumerate}
\end{quote}

Response: We introduce the definition of right-censoring on page 10 of
the revised ms.

\begin{quote}
\begin{enumerate}
\def\labelenumi{\arabic{enumi}.}
\setcounter{enumi}{3}
\tightlist
\item
  Section 1.2 is a bit long. Please shorten this section to improve the
  clarity of the text and reduce redundancy. There is also overlap
  between text in Section 1.2 and the start of Section 4.
\end{enumerate}
\end{quote}

Response: We shortened this section and removed the overlap.

\begin{quote}
\begin{enumerate}
\def\labelenumi{\arabic{enumi}.}
\setcounter{enumi}{4}
\tightlist
\item
  Section 2.3.2. I like the idea of making a connection with
  experimental design, but am having trouble seeing how the discussion
  in this section is specific to EHA and how these comments do not apply
  also to the ``orthodox method''. I think this section could be
  improved by providing more specific distinctions between experimental
  design implications for EHA vs.~the ``orthodox method''.
\end{enumerate}
\end{quote}

Response: We created a new Figure 1 that makes the intended point, and
removed section 2.3.2.

\begin{quote}
\begin{enumerate}
\def\labelenumi{\arabic{enumi}.}
\setcounter{enumi}{5}
\tightlist
\item
  Line 517: What are model weights? Please explain briefly for the
  reader, including how the interpretation differs for loo and waic.
  Also, in the provided reference (Kruz 2023a), I was only able to find
  a discussion of dWAIC, rather than WAIC, so could you please include a
  definition of or reference for dWAIC in your context? Lastly, when
  reporting the model weights in the example in the text, I'd suggest
  including more than 2 digits so that it is clear if the reported
  weights are actually 0 and 1 or if rounding has just made them appear
  so.
\end{enumerate}
\end{quote}

Response: We briefly explain model weights for loo and waic on page 28
of the revised ms. A discussion of WAIC in Kurz (2023a) can be found in
section 4.6.4., so we did not include a reference for dWAIC in the
revised ms. Lastly, we print more digits when reporting the model
weights.

\begin{quote}
\begin{enumerate}
\def\labelenumi{\arabic{enumi}.}
\setcounter{enumi}{6}
\tightlist
\item
  Tutorials 3a and 3b: Rather than simply stating that the frequentist
  models did not converge and resulted in singular fits, it would be
  helpful to provide more discussion. Do you have an idea why the models
  did not converge? Since this is a tutorial paper, more explanation is
  needed. These frequentist examples don't add much to the paper in
  their current form, so I'd either suggest either (a) providing some
  suggestions to the reader of alternative approaches or specific
  strategies that could be used to define a model that will converge or
  (b) removing these examples from the main text and directing the
  reader to the supplement instead.
\end{enumerate}
\end{quote}

Response: We removed the frequentist examples from the main text and
direct the reader to the Tutorials on page 37 of the revised ms. We
mention in Tutorials 3a and 3b that it is common for models to fail to
converge in lme4 when they have a reasonably complex random effects
structure, and we provide a reference.

\begin{quote}
Minor comments: 1. Is ``mean-average comparison'' standard terminology?
This phrase seems redundant to me, since means and averages are the same
thing.
\end{quote}

Response: We removed this term.

\begin{quote}
\begin{enumerate}
\def\labelenumi{\arabic{enumi}.}
\setcounter{enumi}{1}
\tightlist
\item
  Figure 1: In the caption, please state what the small error bars in
  the inset plots are showing.
\end{enumerate}
\end{quote}

Response: The old Figure 1 has been replaced.

\begin{quote}
\begin{enumerate}
\def\labelenumi{\arabic{enumi}.}
\setcounter{enumi}{2}
\tightlist
\item
  Please consider updating all plots so that they are understandable
  when viewed in black and white. In addition to using different colors,
  I'd suggest distinguishing the conditions using shapes (e.g., circles
  vs.~triangles) and different line type (e.g., solid vs.~dashed lines).
\end{enumerate}
\end{quote}

Response: We updated all plots by distinguishing the conditions using
different shapes and/or line types, so that they are understandable when
viewed in black and white.

\begin{quote}
\begin{enumerate}
\def\labelenumi{\arabic{enumi}.}
\setcounter{enumi}{3}
\tightlist
\item
  Figure 2 caption:a. Please provide a reference for the statement
  beginning, ``Because the survival function\ldots{}''.
\end{enumerate}
\end{quote}

Response: Figure 2 has been removed. We only discuss the survival
function in the Supplemental Material (section B) of the revised ms.

\begin{quote}
\begin{enumerate}
\def\labelenumi{\alph{enumi}.}
\setcounter{enumi}{1}
\tightlist
\item
  I am confused by the sentence beginning: ``For example, the high
  hazard of \ldots{}''. Could you please provide some additional
  explanation or a reference for more details on this interpretation?
\end{enumerate}
\end{quote}

Response: We removed this section.

\begin{quote}
\begin{enumerate}
\def\labelenumi{\arabic{enumi}.}
\setcounter{enumi}{4}
\tightlist
\item
  Lines 156-166: Generally, survival probabilities are described as
  probabilities of remaining event-free up until a certain point, rather
  than the probability of an event occurring in the future. Some clarity
  in wording could be used here.
\end{enumerate}
\end{quote}

Response: We follow this definition in section B of the Supplemental
Material.

\begin{quote}
\begin{enumerate}
\def\labelenumi{\arabic{enumi}.}
\setcounter{enumi}{5}
\tightlist
\item
  Please confirm that you have defined all acronyms, abbreviations, and
  functions at their first use in the text. E.g., please define WAIC,
  LOO, P(t), ca(t), SOA, CrI, among others.
\end{enumerate}
\end{quote}

Response: We define all acronyms on their first mention in the revised
ms.

\begin{quote}
\begin{enumerate}
\def\labelenumi{\arabic{enumi}.}
\setcounter{enumi}{6}
\tightlist
\item
  Section 3.1 could be condensed in the main paper (additional details
  could be included in the supplementary material if important). Also,
  please double check your definition of functional programming.
\end{enumerate}
\end{quote}

Response: Section 3 has been removed in the revised ms.

\begin{quote}
\begin{enumerate}
\def\labelenumi{\arabic{enumi}.}
\setcounter{enumi}{7}
\tightlist
\item
  Table 2: It could be helpful to show an example of a trial where an
  event never occurs, to help the reader connect what is written in the
  text (line 325-327) to what appears in the table.
\end{enumerate}
\end{quote}

Response: We added data visualizations in section A of the Supplemental
Material.

\begin{quote}
\begin{enumerate}
\def\labelenumi{\arabic{enumi}.}
\setcounter{enumi}{8}
\tightlist
\item
  Figure 4: Why is there no bin 40?
\end{enumerate}
\end{quote}

Response: Because no responses occurred in the first bin with endpoint
40 ms. However, we added this bin back in Figure 2 (previously 4) of the
revised ms to avoid confusion.

\begin{quote}
\begin{enumerate}
\def\labelenumi{\arabic{enumi}.}
\setcounter{enumi}{9}
\tightlist
\item
  Line 423: The phrase ``early responses'' is confusing as it implies
  temporal ordering of the responses (i.e., early vs.~late trials). I'd
  suggest using ``fast response'' or ``rapid responses'', or some other
  phrase along those lines.
\end{enumerate}
\end{quote}

Response: We replaced early with fast on page 22 of the revised ms.

\begin{quote}
\begin{enumerate}
\def\labelenumi{\arabic{enumi}.}
\setcounter{enumi}{10}
\tightlist
\item
  Is the discussion of reference and index coding (lines 457 -- 464)
  needed? In the introduction, the authors state that knowledge of
  regression is assumed so I am not sure this is necessary to include,
  but I defer to the authors' judgment.
\end{enumerate}
\end{quote}

Response: Index coding is not used a lot in experimental psychology so
we included a short description.

\begin{quote}
\begin{enumerate}
\def\labelenumi{\arabic{enumi}.}
\setcounter{enumi}{11}
\tightlist
\item
  Figure 6: I only see one width of credible interval for each line on
  this plot--- please clarify if you are plotting 80\% or 95\% credible
  intervals.
\end{enumerate}
\end{quote}

Response: We corrected this (Figure 4 in the revised ms.).

\begin{quote}
\begin{enumerate}
\def\labelenumi{\arabic{enumi}.}
\setcounter{enumi}{12}
\tightlist
\item
  Line 536: To me, ``subject-specific'' and ``marginal'' mean the
  opposite things. That is, subject-specific is an effect that is
  conditional on an individual while a ``marginal'' effect has been
  marginalized/averaged over individuals. Could you please clarify your
  use of these terms here?
\end{enumerate}
\end{quote}

Response: We feel there is no real consensus in the literature about the
use of the term marginal. We thus removed this term in the revised ms.
But we kept the term in tutorials 2a and 2b where we base our use of the
terms on this cited reference: Heiss (2021).

\begin{quote}
\begin{enumerate}
\def\labelenumi{\arabic{enumi}.}
\setcounter{enumi}{13}
\tightlist
\item
  Lines 584-587: Since this is a tutorial paper, I'd suggest including
  explanations for why you make changes (a), (b), and (c).
\end{enumerate}
\end{quote}

Response: We now write ``The general process is similar to Tutorial 2a,
except that (a) we use the person-trial data, (b) we use the symmetric
logit link function, and (c) we change the priors (our prior belief is
that conditional accuracy values between 0 and 1 are equally likely).''
on page 33 of the revised ms. The logit is typically used for logistic
regression by experimental psychologists, so we feel that no extra
explanation is needed.

\begin{quote}
\begin{enumerate}
\def\labelenumi{\arabic{enumi}.}
\setcounter{enumi}{14}
\tightlist
\item
  Line 659: What is a ``fully varying effects structure''? Please
  define/explain.
\end{enumerate}
\end{quote}

Response: We explain this on page 24 of the revised ms.: ``We also use
a''keep it maximal'' approach by specifying a full varying (or random)
effects structure. This means that wherever possible we include varying
intercepts and slopes per participant.''

\begin{quote}
\begin{enumerate}
\def\labelenumi{\arabic{enumi}.}
\setcounter{enumi}{15}
\tightlist
\item
  Line 832-833: Please provide a reference for the statement, ``However,
  they require much more data\ldots{}''.
\end{enumerate}
\end{quote}

Response: References are provided on page 8 of the revised ms.

\begin{quote}
\begin{enumerate}
\def\labelenumi{\arabic{enumi}.}
\setcounter{enumi}{16}
\tightlist
\item
  Github code: Instruction 2 of the README states ``open the
  reproducible\_workflow.Rproj'' -- I can't find this R project, should
  it be the ``Tutorial\_EHA.Rproj'' instead?
\end{enumerate}
\end{quote}

Response: Yes, this has been corrected now.

\section{Reviewer: 3}\label{reviewer-3}

\begin{quote}
Comments to the Author This tutorial guides readers through how to
wrangle cognitive reaction time data and address research questions
pertaining to temporal dynamics of cognition using event history
analysis (EHA) and speed/accuracy tradeoff analysis (SAT). Models are
illustrated for hazard functions, and conditional accuracy functions. A
nice feature of this study is that each model is presented in both the
Bayesian and frequentist frameworks. Having both frameworks provides
side-by-side comparisons such as when the Bayesian framework might be
preferred (e.g., when models do not converge in the Frequentist
framework, which is common when modeling multilevel data). The tutorial
also includes a section on simulation and power analysis for planning
experiments. I believe this tutorial could eventually be a useful
contribution to fellow psychological scientists. However, in its current
state, it was difficult to follow/ make my way through (despite existing
expertise in EHA), and also difficult to understand what the main
contributions were above and beyond existing works. Additionally, it was
not clear why the discrete time framework was chosen. The temporal
granularity of reaction time data seems like it would be better suited
for a continuous time framework. Altogether, the tutorial covers
important concepts, but it would benefit from greater accuracy and
clearer definitions of key terms. Carefully removing redundancies,
getting to the tutorial section much quicker (currently not until pg.
17), and being more precise with explanations of technical jargon would
enhance its clarity and accuracy.
\end{quote}

Response: We discuss the main contributions above and beyond existing
works on page 6 of the revised ms. We address the issue of treating time
as discrete or continuous on pages 8-9 of the revised ms. We removed
redundancies, get to the tutorial section quicker (page 12) and included
more precise explanations of technical jargon.

\begin{quote}
Code for the tutorials: I went to the github page and found a folder for
each of the 4 tutorials, which included sub-folders for data, figures,
and tables. It took a minute for me to realize that the actual R code
for each tutorial was in the main folder, not within each sub-folder.
This may also be confusing to readers. Additionally, each of the 4
tutorials, along with their sub-tutorials are provided in separate .Rmd
files. There were several download steps needed to get all the
information. To me, this made the user experience tedious, and made the
tutorials seem disjointed from one another. The authors might consider a
more comprehensive single Rmd file that is neatly organized with an
interactive table of contents (in the knitted html file). To me, this
would make the experience much more enjoyable and useful for readers.
Given that this is a tutorial, it would also be nice if the the
RMarkdowns were somewhat more standalone e.g., by including a brief
description of what a life table is and how to interpret the plot.
\end{quote}

Response: We explain the location of the files and the folder structure
in the README file. In addition we make it clear that a single download
step is possible by either cloning the project, or downloading the ZIP
file. We also included a brief description of a life table and the
definitions of the discrete-time functions in the R Markdown file called
Tutorial\_1a.Rmd. At this point, we also prefer a modular approach to
the tutorials rather than combining them. We appreciate that not
everyone might share this preference, but we think there are many
benefits to a modular workflow in this instance.

\begin{quote}
Contribution of this tutorial: It wasn't clear to me as a reader what
the unique contribution is for this tutorial and what gaps in the
literature it fills beyond existing texts and tutorials. For example, is
this a tutorial that is meant to specifically focus on experimental
psychology cognitive task data -- and that this type of data has special
characteristics/associated research questions that necessitate their own
stand-alone tutorial? If so, it may be helpful to give more detail on
what these specific characteristics tend to be that may make it
different from the many other types of psychology data. Or, is this a
tutorial on discrete time survival analysis that is meant to be broadly
applicable to psychological scientists, and the cognitive reaction time
data is just for illustrative purposes. If so, then much less emphasis
should be placed on the specific data type and more emphasis should be
placed more broadly on types of psychological data/ research questions
this method could be applied to. I think the distinction makes a big
difference in the entire framing of the tutorial, and choosing one or
the other will make the tutorial content much easier for readers to
absorb.
\end{quote}

Response: This tutorial is meant to specifically focus on experimental
psychology RT data and this type of data has special characteristics
that necessitate their own stand-alone tutorial. Therefore, we added
visualizations of the types of time-to-event data that are obtained in
typical RT tasks (detection, discrimination, bistable perception) in
section A of the Supplemental Material.

\begin{quote}
On page 7, it states that there are not tutorials aimed at psychological
time-to-event data. Yet, there are at least a few existing empirical
article examples with accompanying tutorials in psychology (listed
below), and arguably some of the popular textbooks are also geared
toward psychological scientists (e.g., Singer \& Willett). Similarly,
many of the textbooks contain code to accompany their text. Regardless,
even if those are not counted, I still think it will be beneficial for
readers if the authors can make a clear case for what is unique about
psychological time-to-event data that necessitates a stand alone
tutorial/ what gaps in the existing textbooks/empirical
articles/tutorials the present tutorial serves to fill.
\end{quote}

Response: In the revised ms. we frame the unique contribution that this
works adds on page 6: ``\ldots we are not aware of any tutorials that
are aimed specifically at psychological RT (+ accuracy) data, and which
provide worked examples of the key data processing and Bayesian
multilevel regression modelling steps.''

\begin{quote}
Lougheed, J. P., Benson, L., Cole, P. M., \& Ram, N. (2019). Multilevel
survival analysis: Studying the timing of children's recurring
behaviors. Developmental Psychology, 55(1), 53.
\end{quote}

\begin{quote}
Elmer, T., van Duijn, M. A., Ram, N., \& Bringmann, L. F. (2023).
Modeling categorical time-to-event data: The example of social
interaction dynamics captured with event-contingent experience sampling
methods. Psychological Methods.
\end{quote}

\begin{quote}
Mills, M. (2011). The fundamentals of survival and event history
analysis. Introducing Survival Analysis and Event History Analysis.
London: SAGE Publications, 1-17.
\end{quote}

Response: We included these references in the revised ms. on page 6.

\begin{quote}
Organization of the tutorial:The current version of the tutorial is much
longer than a typical AMPPStutorial. There are a few possibilities for
dealing with this. First, it may be that the tutorial could be split
into 2, such as splitting into tutorials 1-3 and tutorial 4 on its own.
Another possibility is that the focus could be primarily on the Bayesian
models, and then leave the frequentist models for either a supplement,
or to only live in the RMarkdown. This possibility may also be
beneficial due to the model convergence issues with at least one of the
frequentist models (which is common with multilevel data and why many
psychological scientists are moving to the Bayesian framework). A third
possibility is that the overall tutorial text could be greatly
streamlined, with major cuts to both the introduction and the discussion
sections.
\end{quote}

Response: We focus primarily on the Bayesian models in the revised ms.,
and keep Tutorial 4. We also made major cuts to both the introduction
and the discussion sections to shorten the ms.

\begin{quote}
Clarity and accuracy: -There are several instances where jargon terms
and key concepts pertinent to survival analysis need to be defined
and/or described in slightly more detail than what is currently
provided. For example, clearer descriptions of censoring and risk sets
(plus what it means to be at risk) should be given in the introduction.
\end{quote}

Response: We introduce the definition of right-censoring on page 10 of
the revised ms. We discuss what it means to be at risk in section A of
the Supplemental Material.

\begin{quote}
Additionally, many key details, such as the EHA equations are relegated
to supplementary materials. In my opinion, any key information that
makes the tutorial more accessible and precise should not be in
supplementary materials.
\end{quote}

Response: We placed the equations for h(t) and ca(t) in Figure 1 of the
revised ms. Beyond Figure 1, we still choose to keep any further
reference of equations to the supplemental material (section B). This is
because we feel that one barrier to uptake of these methods is an
aversion to equations. And by placing them in the supplemental material
we feel that we strike a nice balance between non-technical
accessibility and a fullness of mathemathical explanation for key ideas
and concepts that remain easily accessible to interested readers.

\begin{quote}
There are many abbreviations that were difficult to follow. EHA is
arguably standard, but SAT (for speed/accuracy trade-off) is less
familiar, and may therefore benefit from not being an abbreviation.
Similarly referring to hazard function as h(t), conditional accuracy
function as c(t) was less clear than just saying hazard function and
conditional accuracy function.
\end{quote}

Response: Although SAT is rather standard in cognitive experimental
psychology, we tried to limit the use of abbreviations in main body of
the revised ms.

\begin{quote}
It would be helpful if the research questions that can be addressed by
the methods presented are clearly presented in each section. I think
that currently they are getting lost in the tutorial.
\end{quote}

Response: We now discuss general research questions on page 10 of the
revised ms.

We did not add more information about possible research questions in
each section of Tutorial 2a, because we feel that (a) research questions
are highly dependent on the field of study (perception, behavioral
control, memory), (b) this would make the tutorial even longer, and (c)
we mention in the paper that we do not discuss \textbf{why} you might
perform EHA. In other words, we think that the knowledge of experimental
psychologists on regression modeling should be large enough for them to
link the model structure to possible research questions, once they have
read the paper and worked through the code examples.

\begin{quote}
There are several opportunities to describe the method more precisely.
For example, the discrete time hazard function gives a `conditional'
probability (pg. 9).
\end{quote}

Response: Indeed, this was a big error on our part, which we corrected
in the revised ms. Thank you for pointing this out.

\begin{quote}
Additionally, it may be helpful to use more standard notations such as
h(t\_ij) to indicate the conditional probability that the event occurs
in bin j for individual i, given that the individual had not already
experienced the event in a previous time period (lines 159-160, pg. 9).
\end{quote}

Response: We added indexes for individual and trial in the regression
model equations in section E of the supplemental material. Furthermore,
our new description on pages 8 and 9, together with the new section A in
the supplemental material, should make this issue clear for the reader.

\begin{quote}
Similarly, the survival probability is the probability that the event
occurs after bin j, \emph{given that the individual lhas survived up
until bin j}.
\end{quote}

Response: We define S(t) = P(RT \textgreater{} t) now as ``the
probability that the event does not occur before the endpoint of bin t''
in section B of the Supplemental Material.

\begin{quote}
Throughout the tutorial, reference to individuals in the notation is
missing.
\end{quote}

Response: We added reference to individuals and trials (repeated
measures) in the notation of the model equations in section E of the
Supplemental Material. Furthermore, we feel that the information
provided by our new description on pages 8 and 9, together with the new
section A in the supplemental material, is sufficient for understanding
what we mean without including a reference to individuals (and trials)
in the main text.

\begin{quote}
Another inaccuracy is that in the discrete time framework, it is not an
instantaneous likelihood (line 161, pg. 9).
\end{quote}

Response: Indeed, only continuous-time hazard can be called an
instantaneous likelihood. We replaced the term ``likelihood'' with
``risk''.

\begin{quote}
Introduction: -Section 1.1 -- I had a difficult time understanding this
section as the opening to the tutorial, and the last paragraph seemed
out of place/abrupt. felt too vague to be understandable as the opening
to the tutorial.
\end{quote}

Response: We rewrote this section.

\begin{quote}
Given that the speed-accuracy tradeoff analysis is described as central
to this tutorial, it may benefit from being introduced earlier than pg.
7.
\end{quote}

Response: We introduce conditional accuracy functions already in Figure
1 of the revised ms.

\begin{quote}
It seems that comparing and contrasting Bayesian and frequentist methods
may be outside the scope of a tutorial (pg. 7)
\end{quote}

Response: We agree and focus now on the Bayesian models in the revised
ms.

\begin{quote}
As mentioned above, tutorial 4 -- while related -- could potentially be
its own stand-alone tutorial.
\end{quote}

Response: We decided to keep tutorial 4, as we feel that planning future
studies is an important and difficult step to take in these kinds of
studies, which we wanted to keep close to the rest of the material for
ease of understanding a reasonably complete workflow.

\begin{quote}
Censoring is a key component of survival analysis that is missing from
the introduction
\end{quote}

Response: We introduce the definition of right-censoring on page 10 of
the revised ms.

\begin{quote}
It might be useful to indicate early on the distinctions between
multilevel data, single events, and recurring events -- and that this
particular tutorial is only focused on single events.
\end{quote}

Response: In section A of the Supplemental Material we mention that this
Tutorial does not cover recurrent events, and we visualize the different
types of time-to-event data that are obtained in typical RT tasks
(detection, discrimination, bistable perception tasks). This should make
it clear to the reader that we are dealing with single events that are
repeatedly measured within the same individual.

\begin{quote}
It's not clear why a discrete time framework is used given the granular
temporal resolution of reaction time data -- this would also mean that
the data would not need to be separated into bins. Most textbooks (e.g.,
Allison's) often indicate discrete time methods are common for temporal
scales such as months, years, or decades (not milliseconds). Clearer
justification for the discrete time framework and separating the data
into bins would be helpful.
\end{quote}

Response: We address our choice for discrete-time methods on pages 8-9
of the revised ms.:

``\ldots{} the definition of hazard and the type of models employed
depend on whether one is using continuous or discrete time units. As a
lab, and mainly for practical reasons, we have much more experience
using discrete-time EHA, and that is the approach that we describe and
focus on in this paper. This choice may seem counter-intuitive, given
that RT is typically treated as a continuous variable. However,
continuous forms of EHA require much more data to estimate the
continuous-time hazard (rate) function well (Bloxom, 1984; Luce, 1991;
Van Zandt, 2000). Thus, by trading a bit of temporal resolution for a
lower number of trials, discrete-time methods seem ideal for dealing
with typical psychological RT data sets for which there are less than
\textasciitilde200 trials per condition per participant (Panis, Schmidt,
et al., 2020). Moreover, as indicated by Allison (2010), learning
discrete-time EHA methods first will help in learning continuous-time
methods, so it seems like a good starting point.''

\begin{quote}
Clearer description of the data would be helpful for readers (as well as
only having this description in one place in the tutorial to be more
concise). For example, on pg. 10 it is indicated that 200 trials of 1
experimental condition are simulated. Can the event occur at any time in
those 200 trials? If the event occurs in trial 150, do the remaining 50
trials still occur, or does the time series end for that person? How are
those extra trials after an event occurred handled in the data wrangling
and analysis. Can events occur more than once?
\end{quote}

Response: We think that there is a misunderstanding of what ``trial''
means. We now provide a clear description of the meaning of ``trial''
and of the various types of time-to-data obtained in typical RT tasks in
section A of the Supplemental Material.

\begin{quote}
Section 2.2 (pg. 10, lines 168-169), if describing what `statisticians
and mathematical psychologists say', it would be helpful to provide
supporting references.
\end{quote}

Response: We provide supporting references on pages 9-10 of the revised
ms.

\begin{quote}
The numbered list on pg. 11 contained jargon statements without
explanations. Suggest to simplify and keep the focus on the data you are
using and specific research questions you are using survival analysis as
a tool to address.
\end{quote}

Response: We removed the numbered list in the revised ms.

\begin{quote}
In section 2.3, suggest to lead with the data and research questions of
interest and then show how features of survival analysis make it a good
method to examine the research questions. I also felt that up until now,
the tutorial has been very confusing and hard to follow. It could be
that starting the tutorial at section 2.3 (with a few extra details
filled in), may be an alternative.
\end{quote}

Response: Section 2.3 has been rewritten. The new Figure 1 in the
tutorial directly focuses on the comparison between analysing means
vs.~distributional shapes. We also discuss general research questions on
page 10 of the revised ms.

\begin{quote}
Again, section 2.3.1 (the majority of pg. 12) was unclear and difficult
to follow for someone who already has knowledge of survival analysis,
but not much familiarity with cognitive process/reaction time data.
\end{quote}

Response: Section 2.3.1 has been removed.

\begin{quote}
From what is written, I did not have a clear sense of how temporal
states of cognitive processes and theory development could occur (pg.
14, lines 231-29). These ideas and the research questions are scattered
in many places throughout the tutorial, and are not very
clear/straightforward. It may be helpful to locate them in just one
place in the tutorial and provide a clearer description to guide
readers. Similarly, the statement on pg. 14 (lines 238-239) felt rather
strong without any references, and also out of scope for a tutorial.
\end{quote}

Response: We explain the idea of temporal states of cognitive processes
now on pages 3-4 of the revised ms. and illustrate it in Figure 1.

\begin{quote}
It wasn't clear how section 3.1 added value to the tutorial
\end{quote}

Response: Section 3 has been removed.

\begin{quote}
Section 3.2 -- it would be helpful if there was clarification in the
text on when it would not be possible to include varying intercepts and
slopes (i.e., at least clarifying that random intercepts are a
requirement, but sometimes random slopes are removed based on model
convergence issues -- however, this also begs the question of whether a
model should even be presented if it does not converge when it includes
random slopes\ldots)
\end{quote}

Response: We do not discuss frequentist approaches in the main text
anymore (to save space) and only mention that the reader can consult
Tutorials 3a and 3b. We agree that a model should not be presented if it
does not converge when it includes random slopes. However, to save
space, and because this is an active area of research, we do not attempt
to clarify when it would be possible to include varying intercepts
and/or slopes.

\begin{quote}
Additionally in section 3.2, the text on pg. 16 is vague and does not
give much detail to readers.
\end{quote}

Response: Section 3 has been removed.

\begin{quote}
Overall, the introduction spans across 16 pages, which feels long even
for a standard empirical article, let alone a tutorial. As currently
written, I think many readers will get lost before even getting to the
tutorials section.
\end{quote}

Response: We agree, and restructured and shortened the introduction.

\begin{quote}
Tutorials section: -Rather than only using h(t), and ca(t), suggest to
use hazard function and conditional accuracy function along with their
mathematical notation. Without having presented any equations, using
terms such as h(t) and ca(t) seems odd.
\end{quote}

Response: We try to limit the number of abbreviations used in the main
body of the revised ms.

\begin{quote}
Similarly, P(t) is never defined for readers.
\end{quote}

Response: P(t) was and still is defined only in the Supplementary
Material (section B in the revised version).

\begin{quote}
Section 4.1.1. -- it would be helpful to clarify that the risk set also
pertains to individuals (not just bins)
\end{quote}

Response: We clarify this in section A of the Supplemental Material.

\begin{quote}
The connection between trials and bins could be made clearer (this is in
reference to the text on pg. 18, but I think the clarity would need to
come earlier on in the tutorial when describing the data)
\end{quote}

Response: This connection should be clear now based on section A of the
Supplemental Material, and based on the text on pages 7 and 8.

\begin{quote}
Comprehension of the text on pg. 18 would be enhanced if both right and
left censoring had been clearly introduced previously. Although left
censoring is probably unlikely in the reaction time data used in the
tutorial -- if it is the case that this tutorial is meant for broader
data types in psychology, left censoring would certainly be relevant to
many other use cases.
\end{quote}

Response: We discuss various types of censoring (right, left, random
informative, random uninformative) in section A of the Supplemental
Material.

\begin{quote}
Section 4.1.2 was a clearer data description than had been presented up
until that point in the tutorial. Again, it is suggested that the
manuscript/tutorial could be dramatically reduced in length if
information was more concisely but descriptively written in a single
place (e.g., data descriptions).
\end{quote}

Response: We provide the information on data descriptions in a single
place in the revised ms.

\begin{quote}
Suggest to be more precise with the language on pg. 21 (line 355) --
which data are nested within participants? The trial data? The binned
data?
\end{quote}

Response: We specify ``person-trial data'' in the revised ms.

\begin{quote}
pg. 22 (line 358) the way that censoring is described here is confusing.
\end{quote}

Response: We agree and adjusted the description (see pages 10-11 of the
revised ms.).

\begin{quote}
pg. 22 discusses warning messages and lets readers know these can be
ignored but does not give details as to why. Since this is a didactic
tutorial, it may be helpful to have at least some explanation as to why
the warning can be ignored.
\end{quote}

Response: On line 362 it was stated that ``the warning messages are
generated because some bins have no hazard and ca(t) estimates, and no
error bars. They can thus safely be ignored''. We now removed these
messages from the paper and mention them now only in the tutorial
itself.

\begin{quote}
pg. 22 -- since this is a tutorial, it might also be helpful to unpack
in more detail the section about identifying individuals who may be
guessing etc -- especially since these are very important
considerations.
\end{quote}

Response: We have no more detail to share on this issue. On page 19 of
the revised ms. we now write ``In general, it is important to visually
inspect the functions first for each participant, in order to identify
individuals that may not be following task instructions (e.g., a flat
conditional accuracy function at .5 indicates that someone is just
guessing), outlying individuals, and/or different groups with
qualitatively different behavior.''.

\begin{quote}
section 4.3.1 -- (line 451) it would be helpful if more details is given
on how to set the analysis time window. What counts as ``enough data''?
Given that this is a tutorial, these types of detail are critical for
readers to be able to implement the method in their own work
\end{quote}

Response: We changed this sentence into: ``\ldots{} because the first
few bins typically contain no responses, one has to select an analysis
time window, i.e., a contiguous set of bins for which there is data for
each participant.'' on page 24 of the revised ms.

\begin{quote}
section 4.3.1 (line 453) -- more detail is needed on the cloglog link
function and how it differs from the more commonly known about logit
link.
\end{quote}

Response: A visual comparison between the cloglog and logit link
functions is provided in section C of the Supplemental Material.

\begin{quote}
section 4.3.1 (line 456) -- what goes into the decision about whether to
treat time as a categorical or continuous predictor?
\end{quote}

Response: We added the following information on page 24 of the revised
ms.: ``Third, one has to choose whether to treat TIME (i.e., the time
bin index t) as a categorical or continuous predictor (see also section
E of the Supplemental Material). For example, when you want to know if
cloglog-hazard is changing linearly or quadraticly over time, you should
treat TIME as a continuous predictor. When you are only interested in
the effect of covariates on hazard, you can treat TIME as a categorical
predictor (i.e., fit an intercept for each bin), in which case you can
choose between reference coding and index coding.''

\begin{quote}
section 4.3.3. -- it might be helpful to give a brief indicator up front
to readers about the rationale/purpose of fitting both M0i and M1i,
especially given that no interpretation is given for M0i. Similarly, it
would be helpful to give a bit more detail on the purpose of the
comparison of M0i and M1i (given that this is a didactic tutorial).
\end{quote}

Response: We mention now that model M0i is a baseline or reference
model, to be compared with model M01 (see page 26 of the revised ms.).
We also mention that we only report 2 models in the paper from a larger
set of models that are fitted in the file Tutorial\_2a.Rmd.

\begin{quote}
section 4.3.6. -- same thing here -- since this is a tutorial, at least
brief details on what the purpose of this section is would be helpful.
It might also be helpful here to define what is meant by credible
interval so that more novice readers don't confuse it with a confidence
interval.
\end{quote}

Response: We added the purpose (``To make causal inferences\ldots{}'')
and define a credible interval on page 29 of the revised ms.

\begin{quote}
The term SOA is used on pg. 34, but had never previously been defined or
described.
\end{quote}

Response: Stimulus-onset-asynchrony (SOA) is a well-known concept in
experimental psychology. We now only use the term SOA in the
Supplemental Material.

\begin{quote}
The text on pg. 35 (lines 567-574) seems too in-depth for a tutorial
\end{quote}

Response: We removed this text.

\begin{quote}
pg. 36 -- it would be helpful to give more detail on how to interpret
the logit-ca scale
\end{quote}

Response: We decided not to include more detail because logistic
regression is very familiar to experimental psychologists.

\begin{quote}
section 4.7 -- as previously mentioned, it is possible that tutorial 4
could be a stand alone tutorial. If it's not though, more details are
needed up front to tell readers what the purpose is and why this is
useful.
\end{quote}

Response: We decided to keep tutorial 4, and discuss its usefulness on
pages 45-46 of the revised ms.

\begin{quote}
Discussion section: The Discussion section would benefit from being
shortened (AMPPS guidelines indicate discussions should be a brief
summary of their contents rather than a general discussion).
\end{quote}

Response: We shortened the Discussion section.

\begin{quote}
is `how long' a component of survival analysis -- or is it more accurate
to say `whether and when psychological states occur'?
\end{quote}

Response: It is more accurate to say `whether and when' psychological
states occur. However, based on those results, we can deduce information
about how long a psychological state lasts within a trial. Nevertheless,
we replaced this with ``how effects evolve with increasing waiting
time''.

\begin{quote}
pg. 47 (lines 753-761) -- I had a difficult time understanding this
section
\end{quote}

Response: We rewrote this section.

\begin{quote}
section 5.5 -- unclear that the multi-state model is actually an example
where a multi-state model would be relevant\ldots{} (the conceptual
example given seems like it would be more linked to either a hidden
Markov model or recurring event survival model)
\end{quote}

Response: We removed section 5.5.

\begin{quote}
Figures: -Figure 1 is difficult to follow. It is unclear how the means
are being calculated. For example, it seems like in examples 2 and 3,
the means between conditions should be different. I was also confused in
example 4 because the difference in probability between the two
conditions in time bin 2 seems tiny. I was also confused because it
still looked like there was a big difference in probability between the
two conditions at bin 5. Was bin 6 by chance what was meant instead of
bin 5? It also still looked like there were differences later on such as
in bin 7. Example 5 also left me confused. It looked to me like the
effect started at bin 4 and ended at bin 8. Clarification (and perhaps
simplifying) this figure would be useful.
\end{quote}

Response: We created a new Figure 1.

\begin{quote}
The description for figure 2 is unclear, particularly with respect to
the contrasting knowledge that can be gained from the survival function
and hazard function
\end{quote}

Response: Figure 2 has been removed.

\end{document}
